\documentclass[11pt,letterpaper]{article}

% Packages
\usepackage[utf8]{inputenc}
\usepackage[spanish]{babel}
\usepackage[margin=1in]{geometry}
\usepackage{array}
\usepackage{tabularx}
\usepackage{longtable}
\usepackage{booktabs}
\usepackage{enumitem}

% Document
\begin{document}

% Header
\begin{center}
\textbf{Universidad de Puerto Rico en Humacao}\\
\textbf{Departamento de Matemáticas}\\
\textbf{Guía de Estudio COMP 4098} \quad \textbf{Enero -- Mayo 2026}\\
\end{center}

\begin{tabular}{@{}ll@{}}
\textbf{Profesor:} & \underline{Ollantay Medina}\\
\textbf{Correo Electrónico:} & \underline{ollantay.medina@upr.edu}\\
\textbf{Oficina:} & \underline{CNO-190}\\
\textbf{Horas de oficina:} & \underline{Lu,Mi,Vi: 12:30pm-2:30pm}\\
\end{tabular}

\vspace{0.25cm}

\noindent
\textbf{Texto:}

\noindent
Watt, A., College, D., Eng, N., (2014) ``Database Design -- 2nd Edition''
\vspace{0.25cm}

\noindent
\textbf{Descripción del Curso:}

\noindent
Este curso transiciona de sistemas de bases de datos tradicionales a ingeniería de datos moderna y arquitecturas adaptadas para Ciencia de Datos. Los estudiantes dominarán las arquitecturas que potencian analítica moderna e IA. El currículo contrasta sistemas Transaccionales (OLTP) con motores Analíticos (OLAP), y explora un stack de persistencia políglota—incluyendo PostgreSQL, DuckDB, MongoDB y Bases de Datos Vectoriales—para manejar datos estructurados, semi-estructurados y no estructurados esenciales para aplicaciones de IA como Retrieval-Augmented Generation (RAG).

\vspace{0.25cm}

\noindent
\textbf{Objetivos del Curso:}

\noindent
Al finalizar este curso, los estudiantes podrán:
\begin{enumerate}[leftmargin=0.5cm]
    \item \textbf{Distinguir y Diseñar Sistemas de Datos:} Diseñar soluciones usando el motor de almacenamiento apropiado (Orientado a Filas vs. Orientado a Columnas) basado en requisitos de carga de trabajo (OLTP vs. OLAP).
    \item \textbf{Dominar SQL Analítico:} Ir más allá de CRUD básico para realizar análisis de datos complejos usando Funciones de Ventana, CTEs y agregaciones para reportes de alto rendimiento.
    \item \textbf{Gestionar Diversos Modelos de Datos:} Modelar y consultar efectivamente datos usando almacenes Relacionales (PostgreSQL), de Documentos (MongoDB) y Clave-Valor (Redis).
    \item \textbf{Implementar Capas de Acceso de Datos en Producción:} Implementar Capas de Acceso a Datos (DAL) robustas usando Connection Pooling, DAOs y ORMs para desacoplar la lógica de aplicación de la infraestructura de base de datos.
    \item \textbf{Construir Pipelines de Datos para IA:} Integrar Bases de Datos Vectoriales y Embeddings para soportar búsqueda semántica y construir sistemas de Retrieval-Augmented Generation (RAG) para datos no estructurados.
\end{enumerate}

% Main table
\begin{center}
\begin{tabular}{|p{0.95\textwidth}|}
\hline
\textbf{TEMAS} \\
\hline
\textbf{Módulo 1: Fundamentos Relacionales y SQL Transaccional} (4 semanas) \\
Modelo relacional, llaves e integridad referencial. Modelado ER. Normalización y calidad de datos. SQL transaccional con PostgreSQL (DDL, DML, CRUD). \\
\hline
\textbf{Módulo 2: SQL Analítico y Arquitecturas} (4 semanas) \\
OLTP vs. OLAP y almacenamiento orientado a filas vs. columnas. DuckDB para análisis. Joins complejos, agregaciones, funciones de ventana (Window Functions), CTEs. Laboratorio de rendimiento comparando PostgreSQL y DuckDB. \\
\hline
\textbf{Módulo 3: NoSQL y Modelos de Datos Flexibles} (3 semanas) \\
Teorema CAP y sistemas distribuidos. Bases de datos de documentos (MongoDB) y bases de datos clave-valor (Redis). Modelado y consulta de datos semi-estructurados. \\
\hline
\textbf{Módulo 4: Capa de Acceso a Datos y Arquitectura} (1 semana) \\
Arquitectura de la Capa de Acceso a Datos (DAL). Patrones de implementación: DAO, Repository, ORM. Connection Pooling e independencia de base de datos. \\
\hline
\textbf{Módulo 5: Arquitecturas Modernas e IA} (3 semanas) \\
Datos no estructurados y embeddings vectoriales. Bases de datos vectoriales y búsqueda por similitud. Arquitectura RAG (Retrieval-Augmented Generation). \\
\hline
\textbf{Exposición Proyecto Final: Fecha anunciada por Registraduría} \\
\hline
\end{tabular}
\end{center}

\vspace{0.4cm}

\noindent
\textbf{Curva de Notas:} 100 - 85 (A); 84 - 75 (B); 74 - 60 (C); 59 - 50 (D); 49 -- 0 (F)

\vspace{0.4cm}

\noindent
\textbf{Evaluación:}

\noindent
75\% \quad Tres Exámenes Parciales

\noindent
25\% \quad Proyecto Final

\vspace{0.4cm}

\noindent
\textbf{Reglas:}
\begin{enumerate}[leftmargin=0.5cm]
	\item El curso está catalogado como uno presencial. Se requiere la asistencia regular y puntual a las clases. Asignaciones o pruebas cortas presenciales NO se reponen.
	\item Se utilizará la plataforma de Moodle Institucional como recurso de enseñanza complementario y en	caso de que surja la necesidad, la plataforma de ZOOM para ofrecer las conferencias.
    \item A menos que el profesor lo autorice, NO se permite el uso de teléfonos celulares durante la clase (Cert. \# 1994-95-42 de la Junta Académica de la UPRH).
    \item  Se presume que todo trabajo sometido ha sido hecho por el estudiante. Trabajos entregados que reflejan una infracción evidente a esta regla recibirán cero automáticamente. 
\end{enumerate}

\end{document}
